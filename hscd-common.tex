\usepackage[utf8]{inputenc}
\usepackage[T1]{fontenc}

\usepackage{color,epsfig}
\usepackage{amsmath}
%\usepackage{theorem}
\usepackage{amssymb}
\usepackage{calc}
%\usepackage{subfigure}
\usepackage{times}
\usepackage{listings}
\usepackage{rotating}
\usepackage{ae}
\usepackage{cite}
\usepackage[normalem]{ulem}

% The Postscript fonts do not have a bold symbol font.
% Emulate \boldsymbol via the \pmb{} (poor man's bold) command.
\renewcommand{\boldsymbol}[1]{\pmb{#1}}

\usepackage{verbatim}
%\usepackage{fancyvrb}
%% backward compatibility cruft for verbatim
%\newcommand{\verbatiminput}[1]{\VerbatimInput{#1}}

% Use a small font for the verbatim environment
\makeatletter  % makes '@' an ordinary character
\renewcommand{\verbatim@font}{%
  \ttfamily\small\catcode`\<=\active\catcode`\>=\active%
}
\makeatother   % makes '@' a special symbol again

%\hypersetup{
%  colorlinks = true,
%  linkcolor  = black,
%  anchorcolor= black,
%  citecolor  = black,
%  filecolor  = black,
%  menucolor  = black,
%  pagecolor  = black,
%  urlcolor   = black
%}

%\floatstyle{ruled}
%\floatname{algorithm}{\sffamily\bfseries Algorithm}
%\newfloat{algorithm}{tbp}{lol}%[section]

%\newcommand{\remark}{\tiny� \normalsize}
%\newcommand{\trademark}{\tiny\texttrademark \normalsize}

\usepackage{ifpdf}
\ifpdf
  \newcommand{\graphicPostfix}{pdf}
\else
  \newcommand{\graphicPostfix}{eps}
\fi
